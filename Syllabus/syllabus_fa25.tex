\documentclass[11pt]{article}
\pagestyle{empty}

\setlength{\textheight}{8.5in}
\setlength{\topmargin}{0.5in}
\setlength{\headheight}{0in}
\setlength{\headsep}{0in}
% %% \setlength{\footheight}{0in}
\setlength{\oddsidemargin}{0in}
\setlength{\textwidth}{6.5in}

\usepackage{times}
\usepackage{url}
\usepackage{algorithm}
\usepackage{mathtools}
\usepackage{mathptmx}
\usepackage{amssymb}
\usepackage{color}
\usepackage{graphicx}
\usepackage[colorlinks=true, urlcolor=blue]{hyperref}

\begin{document}
\sloppy 
\begin{center}
\LARGE DS 722\\
\Large Mathematical Foundations of DS and ML \\
\Large\rm Fall 2025\\~\\
\end{center}

\centerline{
\begin{tabular}{|l|r|}
\hline
  & Section A1  \\
 \hline
Meeting Place & CAS 224 \\
  Time & MW 4:30-6:15PM  \\
  Instructor & Prof. Scott Ladenheim  \\
  Office & CDS 1545  \\
  Email & \texttt{saladenh@bu.edu}  \\
  Office hours & W: 3-4PM, F: 10-11AM \\
  Webpage & scottladenheim.com/ \\
\hline
\end{tabular}
}
~\\~\\

~\\~\\
 \begin{minipage}[t]{0.60\textwidth}
 \noindent{\large\bf Teaching Fellows:}  Yuke Zhang
 \begin{itemize}
 \item {\bf Office Hours:} TTh 12:30-1:30PM
 \item {\bf Office Hours Location:} 14th Floor Common Area
 \item {\bf Webpage:} www.bu.edu/cds-faculty/profile/yuke-zhang/
 \end{itemize}
 \end{minipage}
%~\\~\\~\\
%\textbf{Course Assistant (see Piazza for office hours):}  
%\begin{itemize}
%\item TBD
%\end{itemize}

\section*{Overview of the Course}
Mathematical Foundations of Data Science and Machine Learning covers essential mathematical tools and concepts needed for future DS courses. This course is intended for first-year students who need to refresh or expand their mathematical knowledge. The course will guide you through topics on Linear Algebra, Optimization, and Probability and Statistics.

\section*{Course Books}
All course materials are available online or through the BU library.
\begin{itemize}
\item \href{https://www.stat.uchicago.edu/~lekheng/courses/309/books/Trefethen-Bau.pdf}{Numerical Linear Algebra}, Trefethen \& Bau 
\item \href{https://www.math.uci.edu/~qnie/Publications/NumericalOptimization.pdf}{Numerical Optimization} Nocedal \& Wright
\item  \href{library.bu.edu}{All of Statistics: A concise course in statistical inference}, Wasserman
\end{itemize}

\section*{Additional Reading}
The following books are additional useful references.
\begin{itemize}
\item \href{https://mmids-textbook.github.io/index.html}{Mathematical Methods in Data Science (with Python)}, Roch
\item \href{https://mml-book.github.io/book/mml-book.pdf}{Mathematics for Machine Learning}, Deisenroth, Faisal, Ong
\item \href{https://probml.github.io/pml-book/book1.html}{Probabilistic Machine Learning: An Introduction} Murphy
\item Calculus: Early Transcendentals, Stewart
\end{itemize}

\section*{Grading Breakdown}

Final grades will be determined according to the following weights:

\begin{description}
    \item[20\%] Homework assignments (lowest grade dropped)
    \item[10\%] Participation (includes attendance in discussion sections)
    \item[35\%] Midterm exam (in-class)
    \item[35\%] Final exam (take-home)
\end{description}

Grade cutoffs are determined after the course is complete.

\subsection*{Homework Expectations}

Success in the course requires steady, consistent engagement with weekly problem sets. Be sure to allocate regular time for completion. Starting homework the night before it is due will significantly compromise your ability to understand the material and perform well.

\subsection*{Late Homework Policy}

Students are allotted a total of 5 late homework days throughout the semester, with a maximum of 2 days allowed per assignment. Late submissions beyond this policy will not be accepted.

\section*{Discussion Attendance Policy}

Attendance will be recorded weekly during the 12 scheduled discussion sections. Students may miss up to 3 sessions and still receive full participation credit. If more than 3 discussions are missed, participation credit will be prorated based on the percentage of sessions attended.

\subsection*{Exams}

The course includes both a midterm and a final exam. The final exam will be cumulative, covering material from the entire semester. Both exams will be administered as closed-book assessments.

\section*{Piazza}

We will use Piazza as the primary platform for class discussions. It is designed to help you get quick and efficient support from classmates and the teaching staff.

~\\\emph{\textbf{Setup:} Our class Piazza
page is at \url{https://piazza.com/bu/fall2025/ds722}.  If you
registered before the semester started, you should have been automatically enrolled. If you registered for the class after the semester started, go to the above Piazza link and enroll yourself using the Access Code {\bf DS722F25}. If you have any problems, please contact your instructor or TF.}

Please post all course-related questions on Piazza rather than emailing the teaching staff. Piazza will also serve as the main channel for distributing materials—including homework assignments and supplementary resources.

If someone posts a question and you know the answer, feel free to respond. However, \emph{do not} post direct solutions to homework problems. It is acceptable to guide peers by pointing them in the right direction or clarifying misunderstandings, but avoid sharing complete answers.

\section*{Gradescope}

Assignments are submitted and graded on Gradescope
(\url{https://www.gradescope.com/}). If you have any questions
about the grading you receive on Gradescope, please submit a regrade request or contact the instructor or TF.
\\
~\\\emph{\textbf{Setup:} If you registered before the semester started, you should have been automatically enrolled. If you registered for the course after the semester started, go to Gradescope at the link above, and enroll yourself using the entry code {\bf X2K6GB}. If you have any problems, please contact your instructor or TF.}

\section*{Programming Environment}

We will use Python as the language for teaching and for
assignments that require coding. You are expected to know Python and
to use it for all coding assignments.

All registered students have been given accounts on the BU Shared Computing Cluster (SCC). You may use this computational resource for both your homework assignments and discussion sections. Instructions for how to use the SCC are available on Piazza.

\section*{Generative AI}

This course follows the \textbf{CDS Generative AI Assistance (GAIA) Policy}. The policy is here:
\url{https://www.bu.edu/cds-faculty/culture-community/gaia-policy/}.
Please read and familiarize yourself with the policy.  Among other
things, it requires you to fully disclose how you used generative AI
tools such as ChatGPT in your work. 

You are permitted—and encouraged—to use generative AI tools to deepen your understanding of course material. However, these tools should support your learning, not replace it. You are ultimately responsible for developing your own insights and problem-solving abilities.

Keep in mind: all exams will be taken without AI assistance. If you rely too heavily on external tools without mastering the concepts yourself, your performance will likely suffer.

Use AI as a learning aid, not a substitute for thinking.

\section*{Academic Honesty}

Collaborative discussion on homework assignments  is permitted, but copying answers, plagiarizing from books, online sources, or old solutions is strictly prohibited. The work you submit must be entirely your own. All members of the class, including teaching staff, are expected to uphold the 
\href{http://www.bu.edu/dos/policies/student-responsibilities/}{Academic Code of Conduct}.

\subsection*{On Searching for Solutions Online}

While it may be possible to find answers online, consider the following before doing so:

\begin{enumerate}
    \item \textbf{If you're stuck on where to begin:} 
    Contact a TF or post your question on Piazza. Helpful guidance is typically available quickly.

    \item \textbf{If you're seeking to validate your solution:}
    First ask yourself whether there's a way to verify it independently. Doing so strengthens your understanding significantly.

    \item \textbf{If you're short on time and tempted to search for answers:}
    Consider the following:
    \begin{enumerate}
        \item Doing so is intellectually dishonest.
        \item You will need to solve similar problems on the midterm and final exams.
        \item You are allowed to skip one homework assignment without penalty.
    \end{enumerate}
    In such cases, it's better to submit your current work and plan more time for future assignments.
\end{enumerate}

\subsection*{Summary}

If you're struggling, submit what you have, avoid shortcuts, and aim to improve your preparation for upcoming tasks.

\section*{Course Schedule}

The schedule below outlines the planned topics for the semester. Please note that it is tentative and subject to adjustment as the course progresses.

~\\
\small
\begin{centering}
\begin{tabular}{||l|p{3in}|l|l||}
\hline\hline
Date & Topics  & Assigned & Due  \\
\hline\hline
9/3 & 1: Math Foundations for DS &   & \\
\hline
9/8 & 2: Linear Algebra 1 & HW1   & \\
9/10& 3: Linear Algebra 2 &   & \\
\hline
9/15 &  4: Projections and Orthogonality  &  &  \\
9/17& 5: QR Factorization &  &   \\
\hline
9/22& 6: Least Squares   & HW2    &HW1   \\
9/24 & 7: Linear Systems of Equations: LU Factorization   &  & \\
\hline
9/29 & 8: Eigenvalues  &  &   \\ 
10/1 & 9: Singular Value Decomposition & HW3 & HW2  \\
\hline
10/6 & 10: Derivatives 1 &  &  \\
10/8& 11: Derivatives 2  &   &   \\  
\hline 
10/14& 12: Integration &  &    \\
10/15 & 13: Optimization 1  & &HW3\\
\hline
10/20 & Review    &  &\\
10/22 &  Midterm &  &  \\ 
\hline
  
10/27 & 14: Optimization 2: 1st order methods  &HW4 &  \\  
10/29& 15:  Optimization 3: 2nd order methods & & \\
\hline

11/3 & 16:  Probability 1: Probability Introduction  &HW5   &HW4   \\ 
11/5 & 17: Probability 2: Random Variables   &  &  \\ 
\hline

11/10 & 18: Probability 3: Expectation &HW6   & HW5\\  
11/12& 19: Probability 4:  Inequalities and Limit theorems   &  &    \\  
\hline
11/17 & 20: Probability 5: Parameter Estimation and MLE &HW7   &HW6 \\  
11/19& 21: Probability 6: Bayesian Statistics 1   &  &   \\  
\hline
11/24 & 22: Probability 7: Bayesian Statistics 2   &   &HW7  \\
11/26 & Thanksgiving &   &  \\  
\hline

12/1 & 23:  Probability 8: Markov Chains  & Final Exam &  \\
12/3 & 24: Probability 9: Hidden Markov Models  &   &  \\ 
\hline
12/8&  25: Probability 10: Hidden Markov Models  &  &  \\
12/10& TBD &  &  \\
\hline
12/15& & & Final Exam\\
\hline\hline

\end{tabular}\\
\end{centering}

\section*{Course Reading}

~\\
\small
\begin{centering}
\begin{tabular}{||l|p{3in}||}
\hline\hline
Lecture & Reading    \\
\hline\hline
1      &    \\
\hline
2     &   Trefethen \& Bau Ch 1,3  \\
\hline
3      &    \\
\hline
4 &  Trefethen \& Bau Ch 2,6   \\
\hline
5 &  Trefethen \& Bau Ch 7,8,10  \\
\hline
6 &  Trefethen \& Bau  Ch 11     \\
\hline 
7 &  Trefethen \& Bau Ch 20,21 \\
\hline 
8 & Trefethen \& Bau 24, 26    \\
\hline 
9 &   Trefethen \& Bau  Ch 4,5    \\
\hline 
10 &  Deisentroth, Faisal \& Ong Ch 5.1-5.2   \\
\hline 
11 &  Deisentroth, Faisal \& Ong Ch 5.3-5.5    \\
\hline 
12 &  Stewart Ch 5,7,15  \\
\hline 
13 &  Nocedal \& Wright Ch 2   \\
\hline 
14 &    Nocedal \& Wright Ch 3  \\
\hline 
15 & Nocedal \& Wright Ch 3, 6    \\
\hline 
16 & Wasserman: Ch 1 \\
\hline 
17 &  Wasserman Ch 2     \\
\hline 
18 &  Wasserman Ch 3 \\
\hline 
19 & Wasserman Ch 4-5\  \\
\hline 
20 &    Wasserman Ch 9   \\
\hline 
21 &   Wasserman Ch 11    \\
\hline 
22 &     \\
\hline 
23 &  Wasserman Ch 23  \\
\hline 
24 &   \\
\hline 
25 &   \\
\hline
\hline

\end{tabular}\\
\end{centering}

\end{document}

